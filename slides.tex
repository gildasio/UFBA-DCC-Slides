\documentclass[aspectratio=169,xcolor=table]{beamer}
\usepackage[utf8]{inputenc}
\usepackage[T1]{fontenc}
\usepackage{lipsum, lmodern}
\usepackage{csquotes}
\usepackage{xcolor}
\usepackage[portuguese]{babel}
\usetheme{DCC}
\graphicspath{{imgs/}}

\author{Gildásio Júnior}
\title{Template LaTeX seguindo modelo utilizado no DCC/UFBA}
\institute{Universidade Federal da Bahia}
\date{\today}

\begin{document}

\begin{frame}[plain,noframenumbering]
    \titlepage
\end{frame}

\begin{frame}[plain,noframenumbering]{Agenda}
	\tableofcontents
\end{frame}

\section{Introdução}

\begin{frame}{Disclaimer :)}
    Essa "conversão" para o formato LaTeX \cite{mittelbach2004latex} a partir de
    um PPTX foi feita por um aluno na correria para entregar o TCC.

    Portanto sei que há alguns probleminhas, e aceito contribuições, inclusive :)

    O template está bem funcional, utilizei em algumas disciplinas e no TCC ;)
\end{frame}

\section{Seção}

\begin{frame}{Seção}
    As seções servem para já criar a "Agenda"
\end{frame}

\section{Listas}

\begin{frame}{Introdução}
    Podemos fazer listas:
	\begin{itemize}
		\item Primeiro item
		\item Segundo
        \item Em diante...
	\end{itemize}
	\begin{enumerate}
		\item Item 1
        \item Item 2
		\item E esse é?
	\end{enumerate}
\end{frame}

\section{Formatação}

\begin{frame}{Formatação}
    O texto pode estar em \textbf{destaque}, \textit{itálico}, em outras 
    {\color{red}cores}, enfim, como quiser.
\end{frame}

\section{Imagens}

\begin{frame}{Imagens}
    \begin{figure}
        \centering
        \includegraphics[width=0.2\textwidth]{dcc.png}
        \caption{Exemplo de uso de uma imagem}
    \end{figure}
\end{frame}

\section{Referências}
\nocite{*}

\begin{frame}[plain,noframenumbering]{Referências}
    As referências são colocadas no formato bibtex no arquivo refs.bib.

    A opção nocite foi colocada para inserir todas referências do arquivo. Caso
    queira exibir somente as citadas, pode remover essa linha.
\end{frame}

\begin{frame}[plain,noframenumbering,allowframebreaks]
        \frametitle{Referências}
        \bibliographystyle{plain}
        \bibliography{refs}
\end{frame}

\end{document}
